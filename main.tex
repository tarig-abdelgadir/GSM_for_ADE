\documentclass{amsart}
\usepackage[utf8]{inputenc}

\usepackage{mymacros, stmaryrd, tikz, pstricks, pst-node, pst-text, pst-tree, subfigure, epsfig, psfrag}
\usepackage[all]{xy}
\usepackage{float}

\newcommand{\balpha}{\boldsymbol{\alpha}}

\newtheorem{thm}{Theorem}
\newtheorem{cor}[thm]{Corollary}
\newtheorem{lem}[thm]{Lemma}
\newtheorem{prop}[thm]{Proposition}

\theoremstyle{definition}
\newtheorem{defn}[thm]{Definition}
\newtheorem{rmk}[thm]{Remark}
\newtheorem{notn}[thm]{Notation}
\newtheorem{conj}[thm]{Conjecture}
\newtheorem{eg}[thm]{Example}

\title{ADE as GLSMs}
\date{\today}

\begin{document}

\maketitle

\section{Introduction}

\section{Background}\label{sc:background}


\subsection{{\red Adjuctions}}

\subsection{{\red GIT}}

\subsection{Quivers}
A quiver $Q$ is specified by two finite sets $Q_0$ and $Q_1$ together with two maps $h, t \colon Q_1 \rightarrow Q_0$. We call the elements of these sets vertices and arrows, respectively.  The maps $h$ and $t$ indicate the vertices at the head and tail of each arrow.
A nontrivial path in $Q$ is a sequence of arrows $p = a_1 \dotsb a_m$ with $h(a_{k}) = t(a_{k+1})$ for $1 \leq k < m$.  We set $t(p) = t(a_{1})$ and $h(p)= h(a_m)$, the length of $p$ is $m$.  
For each $i \in Q_0$ we have a trivial path $\bfe_i$ where $t(\bfe_i) = h(\bfe_i) = i$.  
The path algebra $\bfk Q$ is the $\bfk$-algebra whose underlying $\bfk$-vector space has a basis consisting of paths in $Q$; the product of two basis elements equals the basis element defined by concatenation of the paths if possible or zero otherwise.
The trivial paths $\bfe_i$ give idempotents of $\bfk Q$.

A representation $\bfx = (V_i, x_a)$ of $Q$ consists of a vector space $V_i$ for each $i \in Q_0$ and a linear map $x_a \colon V_{t(a)} \rightarrow V_{h(a)}$ for each $a \in Q_1$. 
The dimension vector of $\bfx$ is the integer vector $(\dim V_{i})_{i\in Q_0}$.  
A map between representations $\bfx = (V_i, x_a)$ and $\bfx' = (V_i', x_a')$ is a family $\xi_{i} \colon V_i^{\,} \rightarrow V_i'$ for $i \in Q_0$ of linear maps that are compatible with the structure maps, that is $x_a' \circ\xi_{t(a)} = \xi_{h(a)} \circ x_a$ for all $a \in Q_1$.  
With composition defined componentwise, we obtain the abelian category of representations of $Q$ denoted rep$_\bfk(Q)$. 
This category is equivalent to the category $\bfk Q$-mod of finitely generated left-modules over the path algebra.
A relation $r$ on $Q$ is a finite linear combination of paths that share the same head and tail.
A quiver with relations is $Q$ along with a finite set of relations $R$.
A representation of a $(Q,R)$ is a representation of $Q$ for which $x_r=0$ for all $r\in R$.
Representations of $(Q,R)$ correspond to left-modules of $\bfk Q/ \langle R \rangle$.

Given a dimension vector $\bfd$, a $\theta \in \bZ^{Q_0}$ for which $\bfd \cdot \theta=0$ defines a stability notion for representations of $Q$ with dimension vector $\bfd$.
A representation $\bfx$ is $\theta$-semistable if, for every proper, nonzero subrepresentation $\bfx' \subset \bfx$, we have $\sum_{i \in Q_0} \theta_i \cdot \dim(V_i') \geq 0$.  
The notion of $\theta$-stability is obtained by replacing $\geq$ with $>$. 
For a given dimension vector $\bfd =(d_i)_{i \in I} \in \bZ_{Q_0}$, a family of $\theta$-semistable quiver representations over a connected scheme $S$ is a collection of rank $d_i$ locally free sheaves $\cV_i$ together with morphisms $\cV_{t(a)} \rightarrow \cV_{h(a)}$ for every $a\in Q_1$. 
When every $\theta$-semistable representation is $\theta$-stable and the dimension vector is primitive this moduli problem is representable by a smooth scheme $\cM^\theta(Q, \bfd)$, see Proposition 5.3 in \cite{King}.
The scheme $\cM^\theta(Q, \bfd)$ comes with a universal family $(\cV_i, \cV_{t(a)} \rightarrow \cV_{h(a)})_{i \in Q_0, a \in Q_1}$. 
We may define analogous notions for quivers with relations, we will denote the corresponding moduli spaces by $\cM^\theta(Q,R,\bfd)$.

The space $\cM^\theta(Q, \bfd)$ may be expressed as a quotient stack.
Pick vector spaces $V_i$ of dimension $d_i$.
Every point of the affine space
$$\cS:= \bigoplus_{a \in Q_1} \Hom(V_{t(a)}, V_{h(a)})$$
defines a representation of $Q$ with dimension vector $\bfd$.
We write $\cS^\theta$ for the open subvariety of points corresponding to $\theta$-semistable representations.
Change of basis gives a group action of $\GL(\mathbf{d}):= \oplus_{i \in Q_0} \GL(V_i)$ on $\cS$.
The multiplicative group $\bG_m$ diagonally embeds in $\GL(\mathbf{d})$ and acts trivially on $\cS$.
Therefore the action of $\GL(\mathbf{d})$ descends to an action of $\Gamma_\bfd:= \GL(\mathbf{d})/\bG_m$ on $\cS$ and we have $\cM^\theta(Q, \bfd) = \cS^\theta / \Gamma_\bfd.$
Given a set of relations $R$, restricting to the closed set of points $\cS$ whose corresponding quiver representations respect the relations gives a closed subset of $\cR \subset \cS$ and $$\cM^\theta(Q,R,\bfd)= \cR^\theta/\Gamma_\bfd.$$

\subsection{The McKay correspondence}\label{ssc:McKay}
Not only is this an abridged version of a rich story but we also spin it here to motivate what follows.
Fix a finite subgroup $G \subset \SL(2,\bC)$ and let $\rho_0, \ldots, \rho_N$ be its
irreducible representations with $\rho_0$ being the trivial representation.
Set $I=\{0,\ldots,N\}$.
We take $W=\bC^2$ to be the defining representation of $G$.

\subsubsection{The McKay quiver}
The monoidal category $\Rep(G)$ is semi-simple therefore $$\rho_i \otimes W \simeq \bigoplus_{j \in I} c_{ij} \, \rho_j$$
with $c_{ij} =\dim \Hom_{\bC G}(\rho_i \otimes W, \rho_j)$.
The numbers $c_{ij}$ allow us to define the {\em McKay quiver} $Q$: the vertices are indexed by irreducible representations labelled by $I$ with $c_{ij}$  arrows from $i$ to $j$.
The fact that $V$ is a self-dual representation gives that $c_{ij} = c_{ji}$.
It is therefore possible to unambiguously define a graph associated to $Q$ with vertex set $I$ and $c_{ij}$ edges between $i$ and $j$.
This is the McKay graph associated to $G \subset \SL(2,\bC)$.

\begin{theorem}
The McKay graph associated to $G \subset \SL(2,\bC)$ is an affine Dynkin diagram of type $\widetilde{A}\widetilde{D}\widetilde{E}$.
Moreover, the subgraph given by removing the vertex $0$ is the corresponding $ADE$ Dynkin diagram.
\end{theorem}

We identify an arrow $i \rightarrow j$ with a morphism of representations $B_i^j \colon \rho_i \otimes W \rightarrow \rho_j$.
For every irreducible representation this gives us isomorphisms 
\begin{align}
    B_i &\colon \rho_i \otimes W \longrightarrow \bigoplus_{i \rightarrow j} \rho_j \\
    B_{i^\vee} &\colon \rho_i^\vee \otimes W^\vee \longrightarrow \bigoplus_{i \rightarrow j} \rho_j^\vee \otimes \det W^\vee.
\end{align}
We may choose, and do choose, $B_i^j$ so that $B_i^{-1} = \alpha_0^{-1} \circ B_{i^\vee}^T$ up to a fixed isomorphism $\alpha_0 \colon L_W \rightarrow \rho_0$.
The morphisms $B_i^j$ give a natural isomorphism of vector spaces $$\Hom_{\bC G}(\rho_i \otimes W^{\otimes n}, \rho_j) \simeq \langle\, \text{paths of length }n \text{ from }i \text{ to } j\,\rangle.$$
In particular, for a vertex $i$, the one-dimensional space $\Hom_{\bC G} (\rho_i \otimes \det{V}, \rho_i)$ gives, up to $\alpha_0$, a relation $r_i$ of length 2 from $i$ to itself.
The set $R = \{r_i \, |\, i \in I\}$ makes $Q$ into a quiver with relations $(Q, R)$. 
We take $\cA:= \bk Q/ \langle R\rangle$.
On the other hand, the action of $G$ on $W=\bC^2$ induces an action on the symmetric algebra $S(W^\vee)=\bC[x,y]$ which is related to $\cA$ by the following result.

\begin{theorem}
The skew-group algebra $S(W^\vee) \# G$ is Morita equivalent to $\cA$.
\end{theorem}

\subsubsection{The $G$-Hilbert scheme}

The invariant ring of the $G$-action on $S(W^\vee)$ will be denoted $S(W^\vee)^G \subset S(W^\vee)$, it is the centre of $S(W^\vee) \# G$ and is isomorphic to the subring $\bfe_0 \cA \bfe_0 \subset \cA$.
The variety $X:= \Spec(\bfe_0 \cA \bfe_0)$ parametrises the orbits of the $G$-action on $V$ and has a singularity at the $0$-orbit.
The variety $X$ is Gorenstein, that is it is Cohen-Macaulay and its dualizing sheaf $\omega_X$ is invertible.
The $G$-Hilbert scheme, $G$-Hilb(V), parametrises scheme theoretic orbits of $G$ i.e.\ $G$-equivariant ideals $I$ of $S(W^\vee)$ for which $S(W^\vee)/I \simeq \bC G$ as $\bC G$-modules.
The Hilbert-Chow morphism $\pi \colon G\text{-Hilb}(W) \rightarrow X$ takes the zero-locus of $I \subset S(W^\vee)$ to its corresponding $G$-orbit.
Furthermore, $\pi \colon G\text{-Hilb}(W) \rightarrow X$ is a crepant resolution of singualrities, in other words it is a resolution with $\pi^*\omega_W \simeq \omega_Y$.

\begin{theorem}
The inverse image of the singular point $\pi^{-1}(0)$ is a tree of transversely intersecting $\bP^1$s whose intersection graph is dual to the $ADE$ Dynkin diagram corresponding to $G$.
\end{theorem}

%The $G$-Hilbert scheme on $V$ is may be constructed as a moduli of quiver representations of the McKay quiver with relations $(Q,R)$.
Take $\cM^\vartheta:=\cM^\vartheta(Q,R,\bfd)$ for $\vartheta = (-|G|,1, \ldots, 1)$ and $\bfd=(\text{rank}(\rho_i))_{i \in I}$, we have an isomorphism $\cM^\vartheta \simeq G\text{- Hilb}(W)$.

\begin{theorem}
There is an equivalence of triangulated categories $$\Phi \colon D^b(\cM^\theta) \longrightarrow D^b(\cA)$$ that takes the universal bundle $\cV_i$ to the projective $\cA$-module $\cA\bfe_i$.
\end{theorem}

\section*{Notation}
We fix the following choices and notation for the rest of the text.
Fix a finite subgroup $G \subset \SL(2,\bC)$ and let $\rho_0, \ldots, \rho_N$ be its
irreducible representations with $\rho_0$ being the trivial representation.
Set $I=\{0,\ldots,N\}$.
We take $W$ to be the direct sum of representations $\rho_i$ that is isomorphic to defining representation $\bC^2$.
We will write $X:=\Spec(S(W^\vee)^G) = W \GIT G$ and $\bX:=W/G$ for the stack.
The McKay quiver with relations will be denoted $(Q,R)$ and $\cA:= \bk Q/ \langle R\rangle$.
We will write $P_i:= \cA\bfe_i$ for the projective module corresponding to the vertex $i$.

Take $\bfd:=(\text{rank}(\rho_i))_{i \in I}$. 
Pick $V_i$ be vector spaces so that $\dim(V_i)=d_i$ and fix $\theta \in \bZ^{Q_0}$.
We abuse notation and write $W$ for the direct sum of $V_i$ corresponding to the $G$-representation $W$.
Take $$\Gamma:= \bigoplus_{i \in I} \text{GL}(V_i) / \bG_m \quad \text{ and }\quad\cS:=\bigoplus_{a \in Q_1} \Hom(V_{t(a)}, V_{h(a)}).$$
The moduli space $\cM^\theta(Q,\bfd)$ is then $\cS^\theta / \Gamma$.
The relations $R$ give a closed set $\cR \subset \cS$.
We write $\cM:= \cM(Q,R,\bfd)=\cR/\Gamma$ and $\cM^\theta:=\cM^\theta(Q,R,\bfd)=\cR^\theta/\Gamma$.
We will be especially interested in the stability parameter $\vartheta:= (-|G|,1, \ldots, 1) \in \bZ^{Q_0}$.
We let $L_i:= \wedge^{d_i} V_i$ and similarly $L_W:= \wedge^2 W$.
Pick an isomorphism $\bfk \simeq V_0$ once and for all.
%We will use $I^\dagger$ for the natural isomorphism $I^\dagger \colon V^\vee \otimes L_V \rightarrow V$.


\section{The construction}

We define an affine variety $U$ with an action of $\Gamma$, the corresponding stack quotient will be denoted $\cN$.
In subsequent sections, GIT will be used to pick an open substack of $\cN$ isomorphic to $\bX$ and another which isomorphic to $G$-Hilb$(W)$.
We begin by describing $U$ as a closed subset of an affine space $\bA$.

\subsection{The fields}
Our affine space $\bA$ is defined as
\begin{equation*}
    \Hom(V_0,W) \oplus \Bigg(\bigoplus_{0\neq i \in Q_0} \Hom\Big(L_i \otimes L_i,\, \bigotimes_{i \rightarrow j} L_j\Big) \Bigg) \oplus  \Bigg(\bigoplus_{0\neq i \in Q_0} \Hom\Big(V_i \otimes W,\, \bigoplus_{i \rightarrow j} V_j\Big) \Bigg)
\end{equation*}
Points in $\bA$ are given by a tuple $(\bfx,\alpha_i, B_i)_{i \in Q_0}$ as described below:
\begin{align}
\bfx \colon &V_0 \longrightarrow W \label{fd:x} \\
\alpha_i \colon &L_i \otimes L_i \longrightarrow \bigotimes_{i \rightarrow j} L_j \quad \text{for all} \,\, i\neq 0 \label{fd:alpha} \\
 B_i \colon &V_i \otimes W \longrightarrow \bigoplus_{i \rightarrow j} V_j \quad \text{for all} \,\, i\neq 0. \label{fd:B}
\end{align}
Change of basis gives a natural action of $\Gamma$ on $\bA$.

\begin{remark}\label{rm:cartan}
The weights of the $\Gamma$-action on the fields $\alpha_i$ are the columns of minus the Cartan matrix of the corresponding Dynkin diagram.
\end{remark}

In what follows, we manipulate the data $(\bfx, \alpha_i, B_i) \in \bA$ in various ways; the rest of this subsection sets up the notation required.

\begin{notation}
A scalar field of particular importance is $$\alpha_0:= \bigotimes_{0 \neq i \in I} \,\alpha_i^{d_i}: L_W \rightarrow L_0 \otimes L_0.$$
\end{notation}

\begin{notation}
We take $B_0$ to be the unit morphism.
We write $B_i^j$ for the projection of $B_i$ to the $j$-th summand. 
Summing over arrows with head at $i$ we get $$\bigoplus_{j \rightarrow i} B_j^i \colon \bigoplus_{j \rightarrow i} V_j \otimes W \longrightarrow V_i.$$
Furthermore, using some adjuctions, we get a field $$B_{i^\vee}: = \bigoplus_{j \rightarrow i} (B_j^i)^T \colon V_i^\vee \otimes W^\vee \longrightarrow \bigoplus_{i \rightarrow j} V_j^\vee \otimes L_W^\vee.$$
\end{notation}

\begin{notation}
Grouping the linear maps $\alpha_i$ and respectively $B_i$ into big linear maps we write: 
\begin{align*}
\balpha &:= \bigoplus_{i\in Q_0} \alpha_i \colon \bigoplus_{i\in Q_0} L_i \otimes L_i \longrightarrow \bigoplus_{i \in Q_0} \Big( \bigotimes_{i \rightarrow j} L_j \Big) \\
\bfB &:= \bigoplus_{i\in Q_0} B_i \colon \bigoplus_{i\in Q_0} V_i \otimes W \longrightarrow \bigoplus_{a \in Q_1} V_{h(a)}
\end{align*}
Again, using adjuctions appropriately, the morphism $\bfB$ naturally induces a linear map:
$$\widetilde{\bfB} := W \longrightarrow \cS \,\,\Big(:= \bigoplus_{a \in Q_1} \Hom(V_{t(a)}, V_{h(a)})\Big).$$
\end{notation}

Recall that points of $\cS$ define representations of $Q$.
The following lemma then follows from standard linear algebra techniques.

\begin{lemma}\label{lem:Btilde}
For $(\bfx, \balpha, \bfB) \in \bA$ and $a \in Q_1$, the linear map $V_{t(a)} \rightarrow V_{h(a)}$ is $\widetilde{\bfB}(\bfx) \in \cS$ is 
$$V_{t(a)} \xrightarrow{\textup{id} \otimes \bfx} V_{t(a)} \otimes W \xrightarrow{B_{t(a)}} \bigoplus_{t(a) \rightarrow j} V_j \rightarrow V_{h(a)}.$$
\end{lemma}

\subsection{The relations}
Here we describe the equations cutting out $U$ in $\bA$.
We need more notation to do so.

\begin{notation}
We have natural isomorphisms
$$\Omega_r(V) \colon \wedge^{d+r} V \rightarrow \wedge^{d-r} V^\vee \otimes L_U$$ for $V$ a $2d$-dimensional vector space and $r= -d, \ldots, d$.
We will drop the vector space $V$ and the subscript $r$ from $\Omega_r(V)$ when it is clear from the context.

We also abuse notation by writing $\alpha_0^r \alpha_i^{-1} := \alpha_1^{rd_1} \cdots \alpha_i^{rd_i-1} \cdots \alpha_N^{rd_N}$ when $r\geq 1$ even though $\alpha_i$ is not invertible in general.
\end{notation}

\begin{definition}\label{def:equa}
The closed subvariety $U \subset \bA$ is defined as the locus satisfying the following equations:
\begin{align}
    \wedge^{d_i+r} B_i &= \alpha_0^r \alpha_i \circ \Omega_r(V_i \otimes W) \circ \wedge^{d_i-r} B_{i^\vee} \circ \Omega_r(\oplus V_j)^{-1}  \label{eq1:cons2} \\
    \wedge^{d_i} B_i &= \alpha_i \circ \Omega_{0}(V_i \otimes W) \circ \wedge^{d_i} B_{i^\vee} \circ \Omega_{0}(\oplus V_j)^{-1} \label{eq2:cons2} \\
    \alpha_0^r \alpha_i^{-1} \circ \wedge^{d_i-r} B_i &= \Omega_{-r}(V_i \otimes W) \circ \wedge^{d_i+r} B_{i^\vee} \circ \Omega_{-r}(\oplus V_j)^{-1} \label{eq3:cons2} 
\end{align}
for $i\in I$ and $r=1,\ldots,d_i$.
\end{definition}

The equations are a bit of a mouthful, the following lemmas are three different ways of saying almost the same thing but they will help us get a better handle on these equations.

\begin{lemma}\label{lem:symp1}
If $\alpha_0$ is invertible the fields $B_i$ are isomorphisms and the relations imply that they are `symplectic', i.e.
$$B_i^{-1} = \alpha_0^{-1} \circ \Omega_{d_i-1}(V_i \otimes W) \circ B_{i^\vee}^T \circ \Omega_{d_i-1}(\oplus V_j)^{-1}.$$
\end{lemma}

\begin{proof}
The linear map $\alpha_0$ has a one-dimensional domain and codomain, therefore it is invertible if and only if it is non-zero.
So if $\alpha_0$ is invertible then by definition $\alpha_i$ is also invertible for all $i$.
Equation~(\ref{eq1:cons2}) with $r=d_i$ translates to $\det B_i= \alpha_0^{d_i} \alpha_i$.
The inverse of $B_i$ is then given by $(\wedge^{2d_i-1}B_i)^T \otimes (\det B_i)^{-1}$.
Equation~(\ref{eq1:cons2}) with $r=d_i-1$ is $$\wedge^{2d_i-1} B_i = \alpha_0^{d_i-1} \alpha_i \circ \Omega \circ B_{i^\vee} \circ \Omega^{-1}.$$
The result follows.
\end{proof}

\begin{lemma}\label{lem:rela}
For $(\bfx, \balpha, \bfB) \in U$ the following diagram commutes:
\begin{equation}\label{eq:comdiag}
    \xymatrix{V_i \otimes W \ar[r]^-{B_i} \ar[d]^{\textup{id}} & \big(\bigoplus_{i \rightarrow j}  V_j\big) \ar[r]^-{\Omega\,B_{i^\vee}^T \, \Omega} & V_i \otimes W^\vee \ar[d]^{\textup{id}} \\
    V_i \otimes W \ar[r]^-{\textup{id} \otimes \Omega} & V_i \otimes W^\vee \otimes L_W \ar[r]^-{\textup{id} \otimes \alpha_0} & V_i \otimes W^\vee.}
\end{equation}
In particular, the linear map $\widetilde{\bfB}$ factors through the closed set $\cR$ defined by the relations $R$ on $Q$.
\end{lemma}

\begin{proof}
By Lemma~\ref{lem:symp1} above, the diagram commutes on the open subset $\alpha_0 \neq 0$ and hence it commutes everywhere.

The commutativity of this diagram implies that the composite 
\begin{equation}\label{eq:inR}
    V_{i} \xrightarrow{\textup{id} \otimes \bfx} V_{i} \otimes W \xrightarrow{B_{i}} \bigoplus_{i \rightarrow j} V_j \xrightarrow{\Omega\,B_{i^\vee}^T \, \Omega} V_{i} \otimes W^\vee \xrightarrow{\textup{id} \otimes \bfx^\vee} V_i
\end{equation}
%$$V_i \xrightarrow{\textup{id} \otimes \bfx \otimes \bfx} V_i \otimes W \otimes W \xrightarrow{\textup{id} \otimes \det} V_i \otimes L_W \xrightarrow {\alpha_0} V_i$$
coincides with
$$V_i \xrightarrow{\textup{id} \otimes \bfx} V_i \otimes W \xrightarrow{\textup{id} \otimes \Omega} V_i \otimes W^\vee \otimes L_W \xrightarrow{\textup{id} \otimes \alpha_0} V_i \otimes W^\vee \xrightarrow {\textup{id} \otimes \bfx^\vee} V_i$$ 
which is $0$ since $\Omega$ is anti-symmetric.
Composing the linear maps in (\ref{eq:inR}) differently we get $$V_i \longrightarrow \bigoplus_{i \rightarrow j} V_j \longrightarrow V_i$$
is 0 and hence the relation $R_i$ holds.
\end{proof}

\begin{lemma}\label{lem:symp2}
For $(\bfx, \balpha, \bfB) \in U$ the map $\widetilde{\bfB}$ is symplectic, i.e.\ it intertwines the natural symplectic forms on $W$ and $\cS$.
In other word, the following composite coincides with $\Omega \circ \alpha_0$:
\begin{equation}
    W \xrightarrow{\widetilde{\bfB}} \bigoplus_{a \in Q_1} \Hom(V_{t(a)}, V_{h(a)}) \xrightarrow{\Omega} \bigoplus_{a \in Q_1} \Hom(V_{h(a)}, V_{t(a)}) \xrightarrow{\widetilde{\bfB}^T} W^\vee.
\end{equation}
\end{lemma}

\begin{proof}
This follows from tensoring Equation~(\ref{eq:comdiag}) with $V_i^\vee$ and contracting.
\end{proof}

%We unpack Equations~(\ref{eq1:cons2}) and (\ref{eq2:cons2}) to demonstrate that it is homogeneous with respect to the natural $G$ action on our fields:
%Precomposing $\wedge^{d_i+r} B_i$ by the inverse of isomorphism $I^{\dagger}_r(T_i \otimes V)$ and postcomposing $I^{\dagger}_r(\oplus T_j)$ gives a field that differs from $\alpha_i \circ B_{i^\vee}$ by some multiple of $L_V$.
%The factor $\alpha_0^r$ is used to fix this discrepancy.

% \begin{remark}
% With $r=d_i-1$, the relations in (\ref{eq1:cons2}) and (\ref{eq2:cons2}) come agonizingly close to expressing $B_{i^\vee}$ in terms of $B_{i}$ or vice versa but just miss because the fields $\alpha_i$ are not invertible.
% \end{remark}


% that the section $\wedge^{d_i} P_i$ maybe rewritten as $$\wedge^{d_i} P_i \colon \bigoplus_\lambda\, (\bS_\lambda T_i \otimes \bS_{\lambda'} V) \longrightarrow \bigoplus_{\sum_{i\rightarrow j} n_j=d_i} (\wedge^{n_j}T_j)$$
% where $\lambda$ is a partition of $d_i$ with $\lambda_1 \leq 2$ and $\lambda'$ being the partition dual to $\lambda$.
% Moreover, we have natural isomorphisms \begin{equation}
%     I_\lambda \colon \bS_\lambda W \rightarrow \bS_\lambda W^\vee \otimes L_W^2 \quad\quad \text{and} \quad\quad I^\dagger_k \colon \wedge^{k} W \rightarrow \wedge^{n-k} W^\vee \otimes L_W
% \end{equation}
% where $W$ is an $n$-dimensional vector space and $\lambda$ is a partition of $n$ for which $\lambda_1 \leq 2$.
% Precomposing $\wedge^{d_i} P_i$ by the respective $I_\lambda \otimes \text{id}_V$ and postcomposing with the appropriate direct sum of the isomorphisms $I_k^\dagger$ gives a section which is comparable to  $\alpha_i \circ (\wedge^{d_i} P_{i^\vee})$.

\subsection{The stack $\cN$}
This is the main character of the story.

\begin{definition}
The quotient stack $\cN$ is $U/\Gamma$.
Furthermore, for $\theta \in \bZ^{Q_0}$ a character of $\Gamma$, we write $\cN^\theta:= U^\theta/\Gamma$ where $U^\theta$ is the GIT $\theta$-semistable locus.
The standard representations $V_i$ of $\GL(V_i) \subset \Gamma$ give rise to universal bundles $\cV_i$ on $\cN$ while the linear maps $(\bfx, \balpha, \bfB)$ give corresponding universal sections.
\end{definition}

\begin{remark}
The stack $\cN$ may be thought of as the moduli space of functors from a monoidal category $\cC$ to the category of vector spaces.
This idea is pushed to a more general setting in a subsequent work.
\end{remark}

Lemma~\ref{lem:rela} shows that a $\bC$-point of $\cN$ gives rise to one in $\cM$ via $(\bfx, \balpha, \bfB) \mapsto \widetilde{\bfB}(\bfx).$
This works in families too giving a morphism of stacks
\begin{align*}
    & f \colon \cN \longrightarrow \cM \\
    & (\bfx, \balpha, \bfB) \mapsto \widetilde{\bfB}(\bfx).
\end{align*}
This morphism features heavily in what follows.

We finish the section with a lemma that will be helpful when discussing the $\theta$-stability in relation to $f$.

\begin{lemma}\label{lem:Si}
If $\alpha_i=0$ then $S_i$ is a submodule of $\widetilde{\bfB}(\bfx)$.
\end{lemma}

\begin{proof}
If $\alpha_i=0$ then $\wedge^{d_i} B_i=0$ by Equation~\ref{eq2:cons2}.
Therefore the linear map $$V_{i} \xrightarrow{\textup{id} \otimes \bfx} V_{i} \otimes W \xrightarrow{B_{i}} \bigoplus_{i \rightarrow j} V_j$$ is not injective.
The kernel of this composite then gives a submodule of $\widetilde{\bfB}(\bfx)$ concentrated on the vertex $i$ and is therefore a direct sum of copies of $S_i$.
\end{proof}

% \begin{theorem}
% For a generic character $\phi \in \bZ^{Q_0}$ a point $(\bfx, \alpha_i, \bfB)$ is $\phi$-stable if and only if $\widetilde{\bfB}(\bfx)$ is $\phi^+$-polystable and $\alpha_i \neq 0$ for all $i \in I^\phi_-$.
% \end{theorem}

\section{Tannakian duality and the stack $\bX$}

The focus of this section is the open subset $$\cN^\circ:= \{\,(\bfx, \balpha, \bfB) \in \cN \,\,|\,\, \bfB \text{ is invertible} \,\}.$$
We show that it is isomorphic to $\bX$ and that $f \colon \cN^\circ \rightarrow \cM \rightarrow \cM^0=X$ is the coarse moduli space morphism.
We then go on to show that $\cN^\circ = \cN^{-\vartheta}$ for the generic stability parameter $-\vartheta$.

Note that the invertibility of $\bfB$ is equivalent to the invertibility of $\alpha_i$ for all $i \in I$; the determinant of $\bfB$, given by Equation~\ref{eq1:cons2}, is a product of $\alpha_i$'s and has $\alpha_1 \cdots \alpha_N$ as a factor.
Also note that when $\bfB$ is invertible Equation~\ref{eq2:cons2} allow us to express $\alpha_i$ in terms of the field $\bfB$.

Intertwiners $B_i^j \colon \rho_i \otimes W \rightarrow \rho_j$ naturally give a ring isomorphism $\bfe_0 \cA \bfe_0 \simeq S(W^\vee)^G$.
Below we show that our linear maps $B_i^j \colon V_i \otimes W \rightarrow V_j$ do so too.
For a cycle $p$ of length $k$ with $t(p)=h(p)=0$ let $B_0^p$ denote the composite given by successive applications of $\bfB$ below $$V_0 \otimes W^{\otimes k} \longrightarrow \Bigg(\bigoplus_{i \rightarrow j} V_j\Bigg) \otimes W^{\otimes (k-1)} \rightarrow \cdots \rightarrow \bigoplus_{\text{paths } q \text{ length } k} V_{h(q)} \rightarrow V_{h(p)}=V_0.$$

\begin{lemma}\label{lem:e0Ae0}
When $\bfB$ is an isomorphism the ring homomorphism 
\begin{align*}
\bfe_0 \cA \bfe_0 &\longrightarrow S(W^\vee)
\\ p &\mapsto (V_0 \otimes W^{\otimes k} \xrightarrow{B_0^p} V_0)
\end{align*}
factors through $S(W^\vee)^G$ and gives an isomorphism $\bfe_0 \cA \bfe_0 \simeq S(W^\vee)^G$.
\end{lemma}

\begin{proof}
{\red Expand.}
The ring homomorphism factors through $S(W^\vee)^G$ since $\widetilde{\bfB}(\bfy)$ satisfies the relations of the McKay quiver for any $\bfy \in W$.
Since $\bfB$ is an isomorphism, and so is $\bfB^k$, we have that this ring homomorphism is an isomorphism of vector spaces it is therefore a ring isomorphism.
\end{proof}

\begin{theorem}\label{thm:tannaka}
The stack ${\cN}^\circ$ is isomorphic to $\bX$.
Moreover $f \colon {\cN}^\circ \rightarrow \cM^0 = X$ is the coarse moduli space.
\end{theorem}

\begin{proof}
We begin by showing the section $\bfB$ up to the action of $\Gamma$ give the stack $BG$.
Given $\bfB$, we study the associated $\widetilde{\bfB}$.
Taking invariants, we get 
$$W \xrightarrow{\widetilde{\bfB}} \cR \xrightarrow{\nu} \Spec(\bfe_0 \cA \bfe_0) = X.$$
Call this composite $\mu$.
With $G$ acting trivially on the codomain, Lemma~\ref{lem:e0Ae0} gives that $\mu$ is $G$-invariant and seperates $G$-orbits; it is the quotient morphism.
% Two points $\bfx, \bfy \in W$ lie in the same $G$-orbit if and only if $p(\bfx)=p(\bfy)$ for every $p \in S(W^\vee)^G$.
% For $p \in S^k(W^\vee)^G$, we may interpret $p(\bfx)$ as the linear maps $$\bC = V_0 \xrightarrow{\bfx^{\otimes k}} V_0 \otimes W^{\otimes k} \xrightarrow{p} V_0=\bC$$ and similarly for $p(\bfy)$.
% On the other hand, via $S^k(W^\vee)^G \simeq \bfe_0 \cA \bfe_0$, the element $p$ specifies a linear combination of cycles centred at 0 of length $k$.
% Let $\bfB^k$ denote the composite given by successive application of $\bfB$ below $$V_0 \otimes W^{\otimes k} \longrightarrow \Bigg(\bigoplus_{i \rightarrow j} V_j\Bigg) \otimes W^{\otimes (k-1)} \rightarrow \cdots \rightarrow \bigoplus_{\text{paths } q \text{ length } k} V_{h(q)}.$$
% Since $\bfB$ is an isomorphism so is $\bfB^k$.
% By Lemma~\ref{lem:Btilde}, the linear map corresponding to this linear combination in $\widetilde{\bfB}(\bfx)$ is $$V_0 \xrightarrow{\text{id} \otimes \bfx^{\otimes k}} V_0 \otimes W^{\otimes k} \xrightarrow{\bfB^k} \bigoplus_{\text{paths } q \text{ length } k} V_{h(q)} \xrightarrow{p} V_0$$
% respectively.
%The claim follows because $\bfB^k$ is an isomorphism.

We must now consider the effect of the action of $\Gamma$ on $\mu$.
The group $\Gamma$ acts freely on $\cR\setminus \nu^{-1}(0)$ giving an isomorphism $(\cR\setminus \nu^{-1}(0)) / \Gamma \simeq W \GIT G \setminus\{0\}$.
Therefore what remains of the action $\Gamma$ on $\mu$ is the $\GL(W)$-action on the codomain but $\mu$ restricted to $W\setminus \{0\}$ is a $G$-torsor so $\{\bfB\, | \, \bfB \text{ is an isomorphism}\}/\Gamma \simeq BG$.

The data of $\balpha$ is accounted for since we are restricted to the subset where $\bfB$ is an isomorphism.
Adding $\bfx \colon V_0 \rightarrow W$ to the data then gives the result.
\end{proof}

% \begin{proof}
% We unfortunately do this case by case.
% The overriding idea is to start with the representation at the `special' vertex $T_0$ of the McKay quiver and reduce the $P_i$s to a normal form one by one from there.
% The first representation one meets from $T_0$ is $V$ and the next one down is a summand of $V \otimes V$ and so on, therefore in tensor categorical terms this strategy goes through understanding the fusion graph of the faithful representation $V$.

% The $A_n$ case is easy, everything in sight is rank 1 or direct sums of rank 1s.

% For $D_n$, we have that $V=T_2$.
% We pick a bases $\{e_1,e_2\}$ for $T_2$ to help with expressing what follows.
% We take the monomials of degree $n$ with respect to this basis as a basis for $S^n(T_2)$.
% Up to the fixed isomorphism $\bfk \simeq T_0$, we may take $P_0$ is taken to be the identity.
% The section $P_{0^\vee}$ naturally gives a map $P_2^0 \colon T_2 \otimes V \rightarrow T_0$, the constraint (\ref{eq1:cons2}) with $i=0$ and $r=0$ then implies that this factors through $L_V=L_2$.
% A choice of $P_{0^\vee}$ fixes a coset of $\SL(T_2)$ in $\GL(T_2)$.

% We next consider the field $P_2$, semi-simplicity implies that $P_2^1$ is orthogonal to $P_2^0$ and so factors through $S^2(T_2)$.
% The constraint (\ref{eq1:cons2}) with $i=1$ and $r=0$ gives us $P_{1^\vee}$ and hence $P_1$.
% The condition that $P_2^1$ factors through $S^2(T_2)$ then translates into into tracelessness of $P_1$.
% We may then act by $SL(T_2)$ by conjugation on $P_1$ to bring it to the form $\text{diag}(\lambda,-\lambda)$ for $\lambda \in \bfk^\times$.
% %Scaling by $T_1$ we take $\lambda$ so that $\lambda^2=-1$.
% Translating back to $P_2^1$ via the constraints, we get that $P_2^1 \colon S^2(T_2) \rightarrow T_1$ factors through the subspace generated by $e_1 e_2$.
% Semi-simplicity then implies that $P_2^3 \colon S^2(T_2) \rightarrow T_3$ is orthogonal to $P_2^1$ and so factors through the subspace $\{e_1^{\otimes 2}, e_2^{\otimes 2}\}$.
% %We may act on $S^2(T_2) \rightarrow T_3$ by $\GL(T_3)$ to fix $\mu_1=1$ and $\mu_2=1$.

% The constraint (\ref{eq1:cons2}) with $i=2$ and $r=1$ gives $P_{2^\vee}$ up to scalar.
% From this, using some natural identities, we may deduce $P_3^2$ up to a scalar multiple.
% Semi-simplicity then specifies the 2-dimensional subspace of $\Hom(T_3 \otimes T_2, T_4)$ in which the field $P_3^4 \colon T_3 \otimes T_2 \rightarrow T_4$ lives.
% A careful consideration of this subspace gives that $$T_2 \otimes S^2(T_2) \xrightarrow{\text{id} \otimes P_2^3} T_2 \otimes T_3 \xrightarrow{P_3^4} T_4$$ must factor through $S^3(T_2)$ and in particular, for our choices above, through the subspace generated by the tensors $e_1^{\otimes 3}$ and $e_2^{\otimes 3}$.
% We may act on $P_3^4$ by $\GL(T_4)$ to bring it to any form we like.

% We proceed to argue like so all the way up to the last rank two representation $T_{n-2}$ reducing all $P_i$ for $i\leq (n-3)$ to some normal form.
% This in turn, by the constraints, fixes $P_{n-2}^{n-3}$ up to scalar.
% We leave $T_{n-2}$ for now and focus on the rank 1 representations $T_{n-1}$ and $T_n$.
% We fix isomorphisms $P_{n-1}$ and $P_n$ so that the corresponding $P_{n-2}^{n-1}$ and $P_{n-2}^{n}$ are mutually orthogonal and orthogonal to $P_{n-2}^{n-3}$; this is designed to satisfy the semi-simplicity constraint.
% Again examining the subspace in which $P_{n-2}^{n-3}$ lives we may deduce that this orthogonality implies that $$T_2 \otimes S^{(n-3)}(T_2) \xrightarrow{\text{id} \otimes P_{n-3}^{n-2}} T_2 \otimes T_{n-2} \xrightarrow{P_{n-2}^{n-1}} T_{n-1}$$
% factors through the subspace of $S^{(n-2)}(T_2)$ generated by $e_1^{\otimes (n-2)}$ and $e_2^{\otimes (n-2)}$.

% Now we may act by $\SL(T_2)$ to transform $P_{n-1}$ into a form we like while preserving everything else.

% \end{proof}

\begin{corollary}
The stack $\cN^{-\vartheta}$ is isomorphic to $\bX$.
\end{corollary}

\begin{proof}
Take $(\bfx, \balpha, \bfB) \in \cN^{-\vartheta}$, we show that $\alpha_i \neq 0$ for all $i \in I$.
Assume, seeking a contradiction, that $\alpha_i =0$ for some $i \in I$ then by Lemma~\ref{lem:Si} we have that $S_i$ is a subrepresentation of $\widetilde{\bfB}(\bfy)$ for any $\bfy \in W$.
Let $\lambda \in \Gamma$ be the one-parameter subgroup corresponding to $S_i$.
The above then implies that the limit $t \rightarrow 0$ of $\lambda \cdot \widetilde{\bfB}$ exist.
The limit also exists for $\bfx$ since $\lambda$ acts with positive powers on $\bfx$.
The subgroup $\lambda$ acts with positive powers on $\alpha_j$ for all $i\neq j\in I$ and so the corresponding limits exists too.
It remains to check $\alpha_i$ but $\alpha_i=0$ so the desired limit exists there.
We may therefore apply the Hilbert-Mumford criterion with the one-parameter subgroup $\lambda$.
The pairing $\langle \lambda, -\vartheta \rangle$ is negative and hence $(\bfx, \balpha, \bfB)$ is $(-\vartheta)$-unstable yielding our desired contradiction.
% Equation~\ref{eq1:cons2} expresses the determinant of $B_i$, and hence $\bfB$, in terms of the scalars $\alpha_i$.
% Furthermore, we have that the weight of $\bfx \colon V_0 \rightarrow W$ is opposite to that of $\alpha_0 \colon L_W \rightarrow L_0$ and so setting $\bfx=0$ does not alter the $\phi$-stability of a point $(\bfx, \alpha_i, \bfB)$.
% Therefore the determinant of any semi-invariants with weight $n \phi$, for $n>0$, must contain the scalar field $\otimes_{i\in I} \, \alpha_i$ as a factor.
% Therefore $\otimes_{i\in I} \, \alpha_i \neq 0$ for any $\phi$-stable point of $\cM$ which implies that $\alpha_i$s are are invertible and hence also $\bfB$.
% The corollary may also be be deduced from Lemma~\ref{lem:Si}.
\end{proof}

\begin{remark}
For a point $(\bfx, \balpha, \bfB) \in \cN^\circ$, the proof of Theorem~\ref{thm:tannaka} shows that the data in $\bfB$ is equivalent to that of a monoidal fibre functor $\Rep(G) \rightarrow \Vect$.
\end{remark}

\section{The $G$-Hilbert Scheme}
In this section we show that the restriction of $f \colon \cN \rightarrow \cM$ to $\cN^\vartheta$ factors through $\cM^\vartheta$.
We go on to construct an inverse $g \colon \cM^\vartheta \rightarrow \cN^\vartheta$ to this restriction of $f$ showing it is an isomorphism.

% Let $\cN^\theta$ denote the quotient of $\theta$-stable fields up to the constraints and $\Gamma$-action.
% We construct morphisms $f \colon \cN^\theta \rightarrow \cM^\theta$ and $g \colon \cM^\theta \rightarrow \cN^\theta$ and show that they are mutual inverses.

\begin{lemma}
The morphism $f \colon \cN \rightarrow \cM$ restricted to $\cN^\vartheta$ factors through $\cM^\vartheta$.
\end{lemma}

\begin{proof}
For $(\bfx, \balpha, \bfB) \in \cN^\vartheta$, we show that $S_i$ can not be a quotient of $\widetilde{\bfB}(\bfx)$. This implies $\vartheta$-stability since any destabilizing quotient must have a simple as a quotient module {\red maybe say some more here to make it crystal clear}.
Assume seeking a contradiction that $S_i$ is a quotient of $\widetilde{\bfB}(\bfx)$.
This is equivalent to the vanishing of the $d_i$-th wedge of the linear map
$$\bigoplus_{i \rightarrow j} V_j \xrightarrow{B_{i^\vee}} V_i \otimes W^\vee \xrightarrow{\text{id} \otimes \bfx} V_i.$$

Equation~\ref{eq2:cons2} then implies that $S_i$ also a submodule of $\widetilde{\bfB}(\bfx)$ since the $d_i$-th wedge of 
$$V_i \xrightarrow{\text{id} \otimes \bfx} V_i \otimes W \xrightarrow{B_i} \bigoplus_{i \rightarrow j} V_j$$
vanishes too.
In fact, Equation~\ref{eq2:cons2} implies that $S_i$ is a direct summand of $\widetilde{\bfB}(\bfx)$.
Therefore evaluating 
\begin{equation}\label{eq:limB}
W \xrightarrow{\widetilde{\bfB}} \cR \longrightarrow \Big(\bigoplus_{i \rightarrow j} \Hom(S_i,V_j) \oplus \bigoplus_{j \rightarrow i} \Hom(V_j, S_i)\Big)
\end{equation}
on $\bfx$ yields 0.
Since, by an argument similar to Lemma~\ref{lem:symp2}, the composite above is symplectic and $\dim W=2$ this is zero on all of $W$.

The quotient $S_i$ of $\widetilde{\bfB}(\bfx)$ by King's argument corresponds to a 1-parameter subgroup $\lambda$ of $\Gamma$ that is the identity on the summands $\GL(V_j)$ for all $j\neq i$ and has negative weight on $\GL(V_i)$. 
Now $\langle \lambda, \vartheta\rangle <0$, therefore $\lambda$ would destabilise $(\bfx, \balpha, \bfB)$ if $\lim_{t \rightarrow 0} \, \lambda(t)\cdot (\bfx, \balpha, \bfB)$ was to exist.
The summand $\GL(V_i)$ acts on $\bfx$, $\alpha_i$ and $\alpha_k$ for $k \in Q_0$ not adjacent to $i$ with non-positive weight and so the corresponding limits exist for $\lambda$.
The vanishing of the linear map in (\ref{eq:limB}) implies that $\lim_{t \rightarrow 0}\, \lambda(t)\cdot \widetilde{\bfB}$ exists.
To reach our desired contradiction, it remains to show that the limit exists for $j$ adjacent to $i$; this is the case only when $\alpha_j=0$.

Assume $\alpha_j \neq 0$.
We have that the following sequence is not exact since $i$ is adjacent to $j$ and $S_i$ is a summand of $\widetilde{\bfB}(\bfx)$
$$0 \longrightarrow V_j \longrightarrow \bigoplus_{j \rightarrow k} V_k \longrightarrow V_j \longrightarrow 0.$$
Equation~\ref{eq2:cons2} and $\alpha_j \neq 0$ means that it is not exact on either side: if it is not exact on the left it is not on the right and vice versa.
Therefore $S_j$ is also a quotient in $\widetilde{\bfB}(\bfx)$.
We leave it for the reader to check that this is compatible with $S_i$, that is that the argument implies that $S_i \oplus S_j$ is a summand of $\widetilde{\bfB}(\bfx)$.
The 1-parameter subgroup corresponding to $S_i \oplus S_j$ then either has a limit and is destabilising or $S_i \oplus S_j$ is extendable to a bigger summand as above.
\end{proof}

\subsection{The inverse morphism}

We define a morphism  $g \colon \cM^\vartheta \rightarrow \bA/\Gamma$.
We show that $g$ factors through $\cN^\vartheta$ and is hence a candidate for the inverse of $f \colon \cN^\vartheta \rightarrow \cM^\vartheta$.
We then go on to prove that it indeed is $f^{-1}$.

Take $(\cV_i, v_a)$ to be the universal family on $\cM^\vartheta$.
To define $g$ it suffices to describe the pullback of the universal sections $\bfx$, $\bfB$ and $\balpha$ on the vector bundles $\cV_i$.

\subsubsection{The field \bfx}

The representation $W$ either corresponds to a single vertex if irreducible or a pair of vertices if not.
The arrows from vertex $0$ to vertices corresponding to $W$ gives us a tautological section $\cV_0 \rightarrow \cW$ we take that to be the $\bfx$ of Equation~(\ref{fd:x}).

\subsubsection{The field $\bfB$}
In this construction, will use the isomorphism $\End_{\cM^\theta}(\oplus_i V_i) \simeq \End^{\text{op}}_\cA(\oplus_i P_i)$.
We have that the isomorphism of $G$-representations $\rho_i \otimes W \rightarrow \oplus_j\, \rho_j$ induces a morphism of $G$-representations $$\underline{\Hom}_G(W,\rho_0) \longrightarrow \underline{\Hom}_G(\oplus_j \, \rho_j, \rho_i).$$
This, in turn, gives a linear map of Hom-spaces of the corresponding projective modules $P_i$ and a suggestively named linear map $H^0(\widetilde{\bfB})$:
\begin{equation}\label{eq:homsp}
    H^0(\widetilde{\bfB}) \colon {\Hom}(\cV_0, \cW) \longrightarrow {\Hom}(\cV_i, \oplus_j \, \cV_j).
\end{equation}

Next we use $\vartheta$-stability of the universal family $(\cV_i, v_a)$.
We have that $\vartheta_0<0$ and $\vartheta_i>0$ for all other $i$ therefore the evaluation morphism $$\text{ev} \colon \cV_0 \otimes_\bC {\Hom}(\cV_0, \cW) \longrightarrow \cW$$ is surjective.
Combining this with Equation (\ref{eq:homsp}) above we get a composite
$$\cV_0 \otimes_\bC {\Hom}(\cV_0, \cW) \longrightarrow \cV_0 \otimes_\bC {\Hom}(\cV_i, \oplus_j \, \cV_j) \longrightarrow\ \cV_i^\vee \otimes \Big(\bigoplus_{i \rightarrow j}\cV_j\Big)$$
where the second factor is also an evaluation map.
We claim that the kernel of ev is zero under this composite which yields a section 
$$\widetilde{B}_i \colon \cW \longrightarrow \cV_i^\vee \otimes \Big(\bigoplus_{i \rightarrow j}\cV_j\Big).$$
Summing over the vertices $i$ gives us $\widetilde{\bfB}$ and hence $\bfB$.
It remains to prove the claim.

\begin{proof}[Proof of claim]
Take $\cK$ in the kernel of $\ev$.
We may assume that this is a universal bundle since they generate $D^b(\cM^\vartheta)$ so take $\cK=\cV_k$.

We now translate the statement in terms of Hom-spaces of the corresponding projective modules and use the monoidal structure induced by that of $G$-representations there.
The fact that $\cK$ is in the kernel is given by an element $p$ in the kernel of 
\begin{equation}
    \Hom(P_k, P_0) \otimes_\bC \Hom(P_0, P_W) \longrightarrow \Hom(P_k, P_W).
\end{equation}
That is, $p$ corresponds to a linear combination of paths from $k$ to $0$ and a linear combination of paths from $0$ to $W$ for which the product is in the ideal of relations.
The following diagram is commutative:
\begin{equation*}
    \xymatrix{\Hom(P_k, P_0) \otimes_\bC \Hom(P_0, P_W) \ar[r] \ar[d]^{- \otimes_G P_i} & \Hom(P_k, P_W) \ar[d]^{- \otimes_G P_i} \\
    \Hom(P_k \otimes_G P_i, P_i) \otimes_\bC \Hom(P_i, P_\cW \otimes_G P_i) \ar[r] & \Hom(P_k \otimes_G P_i, P_\cW \otimes_G P_i).}
\end{equation*}
The module $P_\cW \otimes_G P_i$ is $\oplus_j P_j$, similarly $P_k \otimes_G P_i$ is isomorphic to projective module $P_l$. 
Mapping $p$ through the vertical linear map $- \otimes_G P_i$ gives an element of $\Hom(P_l, P_i) \otimes_\bC \Hom(P_i, \oplus_j P_j)$.
The commutativity of the diagram implies that this gives rise to an element of the kernel of the evaluation map $$\cV_i \otimes_\bC \Hom(\cV_i, \oplus_j \cV_j) \rightarrow \oplus_j \cV_j$$
Adjunction completes the proof {\red lazy}.
\end{proof}


\subsubsection{The field $\balpha$}\label{sssc:alpha}

Take $0 \neq i \in Q_0$.
Since our family is $\vartheta$-stable and $\vartheta_i>0$, the universal section corresponding to arrows whose heads are $i$ is surjective, i.e.\ the following is exact:
\begin{equation}\label{eq:k}
    \bigoplus_{i \rightarrow j} \cV_j \longrightarrow \cV_i \longrightarrow 0.
\end{equation}
Take $\cU$ to be its kernel.
The relation $r_i \in R$ give us a complex
$$\cV_i \longrightarrow \bigoplus_{i \rightarrow j} \cV_j \longrightarrow \cV_i \longrightarrow 0$$
and hence a morphism $\cV_i \rightarrow \cU$.
We take $\alpha_i$ to be $\wedge^{d_i} \cV_i \rightarrow \wedge^{d_i}\cU.$

% \begin{remark}
% The kernel $\fK$ is quasi isomorphic to the complex in (\ref{eq:k}).
% As with any Schur functor, the top wedge of a complex maybe expressed as a complex.
% We may equivalently use this expressing of the top wedge along with the constraints (\ref{eq1:cons2}) and (\ref{eq2:cons2}) to define $\alpha_i$.
% \end{remark}

\subsubsection{The relations and conclusion}

The above construction defines a morphism $g \colon \cM^\vartheta \rightarrow \bA/\Gamma$ here we show that it in fact factors through $\cN^\vartheta$.
Then prove it is $f^{-1}$.

Theorem~\ref{thm:tannaka} gives that $\cM^\vartheta$ and $\cN^\vartheta$ are birational via $f \colon \cN \rightarrow \cM$.
To deduce that the relations are satisfied we first show that $f \circ g \colon \cM^\vartheta \rightarrow \cM^\vartheta$ is the identity on the open subset of points of $\cM^\vartheta$ corresponding to free orbits of the $G$-action on $W$.

\begin{theorem}\label{thm:fg}
The morphism $f \circ g \colon \cM^{\vartheta} \longrightarrow \cS/\Gamma$ factors through $\cM^\vartheta$ and is the identity on it.
\end{theorem}

\begin{proof}
Take $\bfe_\bfx \colon \cV_0 \rightarrow \cV_0 \otimes_\bC \Hom(\cV_0, \cW)$ to be the section taking $\cV_0$ to the component corresponding to the arrow(s) with tail at the vertex $0$.
The right hand column of the following diagram is how $f$ is defined while the left hand column is how $g$ is defined.
\begin{equation}\label{eq:gf}
    \xymatrix{\cV_0 \ar[r]^{\text{id}} \ar[d]^{\bfe_\bfx} & \cV_0 \ar[d]^\bfx \\
    \cV_0 \otimes_\bC \Hom(\cV_0, \cW) \ar[r]^-{\text{ev}} \ar[d]^{\text{id} \otimes H^0({\widetilde{\bfB}})} & \cW \ar[d]^{\widetilde{\bfB}} \\
    \cV_0 \otimes_\bC \Hom(\cV_i, \oplus_j \cV_j) \ar[r]^-{\text{ev}} & \cV_i^\vee \otimes \Big(\bigoplus_{i \rightarrow j}\cV_j\Big).
    }
\end{equation}
The commutativity of the diagram then gives the result.
\end{proof}

\begin{corollary}
The morphism $g \colon \cM^\vartheta \rightarrow \bA/\Gamma$ factors through the closed subset $\cN \subset \bA/\Gamma$.
\end{corollary}

\begin{proof}
Theorem~\ref{thm:fg} gives that $g$ is a left inverse to the birational map $f$; it is therefore a two-sided inverse.
The restriction of $g$ to a dense open subset of $\cM^\vartheta$ factors through the closed subset $\cN \subset \cA/\Gamma$ which implies unrestricted morphism $g$ does too.
\end{proof}

\begin{corollary}
The morphism $g \colon \cM^\vartheta \longrightarrow \cN$ factors through the open subset $\cN^\vartheta$.
\end{corollary}

\begin{proof}
The $\vartheta$-stability of a point in $\cM^\vartheta$ is guaranteed by the non-vanishing of section of a power of $\otimes_{i \in Q_0} L_i$.
We may construct such a section on a point $(\bfx, \bfB, \balpha)$ in image by taking the corresponding section on $\widetilde{\bfB}(\bfx)$.
This section remains non-zero by Theorem~\ref{thm:fg}.
\end{proof}

We now have a morphism $g \colon \cM^\vartheta \rightarrow \cN^\vartheta$ and so a candidate for the inverse of $f$.

\begin{theorem}\label{thm:ghilb}
The morphism $g \circ f \colon \cN^\vartheta \rightarrow \cN^\vartheta$ is the identity.
\end{theorem}

\begin{proof}
%Theorem~\ref{thm:fg} gives that $f \circ g$ is the identity, it remains to show that $g \circ f \colon \cN^\vartheta \rightarrow \cN^\vartheta$ is too.
Take $\cV_i$ with sections $(\bfx, \bfb, \balpha)$ to be the universal family on $\cN^\vartheta$.
The vector bundles remain unchanged by $g \circ f$ and so does the section $\bfx$.
The proof of Theorem~\ref{thm:fg}, in particular the bottom half of the commutative diagram (\ref{eq:gf}), shows that the same is true for $\bfB$.

We now focus on the sections $\alpha_i$.
The $d_i$-th wedge of the complex
$$\bigoplus_{j \rightarrow i} \cV_j \xrightarrow{(B_{i^\vee} \,\circ\, \bfx)^T} \cV_i$$
is the complex
\begin{equation}\label{eq:dthwedge}
    \bigwedge^{d_i} \Bigg(\bigoplus_{j \rightarrow i} \cV_j \Bigg) \rightarrow \bigwedge^{d_i-1} \Bigg(\bigoplus_{j \rightarrow i} \cV_j \Bigg) \otimes \cV_i \rightarrow \cdots \rightarrow S^{d_i}(\cV_i).
\end{equation}
In Subsubsection~\ref{sssc:alpha} the section $\alpha_i$ on points in the image of $g$ is defined as the $d_i$-th wedge of the morphism of complexes
\begin{equation*}
    \xymatrix{\cV_i \ar[rr] \ar[d]^{B_i \,\circ\, \bfx} && 0 \ar[d] \\
    \bigoplus_{j \rightarrow i} \cV_j \ar[rr]^-{(B_{i^\vee} \,\circ\, \bfx)^T} && \cV_i.}
\end{equation*}
That is, $\alpha_i$ is given by the composite 
\begin{equation}\label{eq:alpha}
    L_i \xrightarrow{\bfx^{d_i}} L_i \otimes S^{d_i}(\cW) \xrightarrow{\wedge^{d_i} B_{i}} 
%\bigotimes_{i \rightarrow j} L_j  \otimes \bigwedge^{d_i} \Bigg(\bigoplus_{j \rightarrow i} \cV_j \Bigg) = 
\bigwedge^{d_i} \Bigg(\bigoplus_{j \rightarrow i} \cV_j \Bigg).
\end{equation}

By Equation~\ref{eq2:cons2} we may rewrite (\ref{eq:alpha}) as 
$$L_i \xrightarrow{\alpha_i} \bigotimes_{i \rightarrow j} L_j  \otimes L_i^\vee \xrightarrow{\bfx^{d_i}} \bigotimes_{i \rightarrow j} L_j \otimes L_i^\vee \otimes S^{d_i}(\cW) \xrightarrow{\Omega \,\circ\, (\wedge^{d_i} B_{i^\vee}) \,\circ\, \Omega} 
%\bigotimes_{i \rightarrow j} L_j  \otimes \bigwedge^{d_i} \Bigg(\bigoplus_{j \rightarrow i} \cV_j \Bigg) = 
\bigwedge^{d_i} \Bigg(\bigoplus_{j \rightarrow i} \cV_j \Bigg)$$
Now our family $(\bfx, \bfB, \balpha)$ is $\vartheta$-stable therefore the section $(B_{i^\vee} \circ \bfx)^T$ is surjective and the composite $$\bigotimes_{i \rightarrow j} L_j  \otimes L_i^\vee \xrightarrow{\bfx^{d_i}} \bigotimes_{i \rightarrow j} L_j \otimes L_i^\vee \otimes S^{d_i}(\cW) \xrightarrow{\Omega \,\circ\, (\wedge^{d_i} B_{i^\vee}) \,\circ\, \Omega} 
%\bigotimes_{i \rightarrow j} L_j  \otimes \bigwedge^{d_i} \Bigg(\bigoplus_{j \rightarrow i} \cV_j \Bigg) = 
\bigwedge^{d_i} \Bigg(\bigoplus_{j \rightarrow i} \cV_j \Bigg)$$
gives an isomorphism of $\bigotimes_{i \rightarrow j} L_j  \otimes L_i^\vee$ with the complex in (\ref{eq:dthwedge}).
This completes the proof.
\end{proof}

\begin{corollary}
The morphism $f \colon \cN^\vartheta \rightarrow \cM^\vartheta$ is an isomorphism.
In particular, $\cN^\vartheta \simeq G\textup{-Hilb}(W)$.
\end{corollary}

\section{Examples}

% This section will serve both as a demonstration that the above GIT problem maybe expressed as an affine space (linear) modulo the action of a reductive group and as an examples section.
% That is, we will demonstrate this case-by-case.
% The idea is that $\bfB$ gives a symplectic matrix when invertible.
% Up to scalars, the manifold of symplectic matrices maybe thought of as a flag variety which is an affine spaces modulo the action of a reductive group.

% Fix a vertex $i \in Q_0$ and consider the linear map $B_i \oplus B_{i^\vee}$.
% The relations imply that this a symplecto-morphism.
% In particular, given a choice of bases the rows of $B_i$ are orthogonal with respect to the symplectic forms in question.
% They are therefore given by a complete flag.

\subsection{Type A}

This is classical.

\subsection{Type D}
Here, $W=V_2$. 
As before, starting at vertex $0$: 
\begin{align*}
    &B_0=\text{id}_{V_2} \colon V_2 \longrightarrow V_2 \\
    &B_{0^\vee} = \text{id} \otimes \alpha_0 \colon V_2^\vee \longrightarrow V_2^\vee \otimes L_2^\vee.
\end{align*}

The data of $B_{0^\vee}$ is equivalent to $B_2^0$ and so specifies the first morphism in the flag $V_2 \otimes V_2 \rightarrow S^2(V_2)$.
To fix $B_2$ and therefore $B_2^\vee$ it suffices to pick fix a flag 
$$V_2 \otimes V_2 \rightarrow S^2(V_2) \rightarrow V_3.$$

At vertex $3$ we need a flag $V_2 \otimes V_3 \rightarrow V_2$ but that is already fixed by the above data.
We therefore have $B_3$.
Eventually we arrive at the vertex ${n-2}$, then we take a hom $V_2 \otimes V_{n-2} / V_{n-3} \rightarrow V_{n-1}$ which completes the picture.


\bibliographystyle{amsplain}
\bibliography{references}

\end{document}